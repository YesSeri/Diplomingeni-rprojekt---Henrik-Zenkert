
\section{Context, Gap, Innovation and Evaluation OR Abstract?}

DBC Digital operates a fleet of servers specified, built, and deployed
using Nix files stored in a Git repository. They want to monitor build
provenance and the system state, to ensure each server matches its
intended configuration with only expected differences. 
Currently, DBC Digital maps Nix store paths to Git commits in a CSV file, and uses a PHP script to query servers systems state and display the data. This approach provides some traceability but is slow, limited in
interactivity, requires SSH access to all servers (a security risk), and
integrates poorly with existing systems.

This project will build a secure and integrated system for monitoring and
maintaining mappings between Nix store paths and Git revisions, with
history, across a fleet of servers.
At a minimum, the system should match the PHP script's functionality, while offering improved security and performance. Stretch goals include
making the tool easy to deploy in a plug-and-play manner, ensuring it is user-friendly and well-documented as an open-source project, and further enhancing usability.

\section{Technical and Theoretical Background}
In modern distributed infrastructures, reproducibility and traceability are essential to ensure reliable operation across large fleets of servers. This project addresses that challenge in the context of NixOS-based systems deployed across multiple data centres.

The goal is to design and implement a host overview system that establishes a reliable mapping between Nix store paths and the git revisions from which they were built. This mapping is fundamental for two reasons:

1. Cluster consistency: Even though each server’s current-system path is unique, the mapping allows us to verify whether all nodes in a cluster correspond to the same git commit. This ensures that a distributed application is running the same software version across data centres.

2. Reproducibility verification: If multiple git commits produce identical nix store paths, we can confirm that changes in the source repository did not affect the system image of a given node. This provides strong guarantees about which changes are operationally relevant.

Embedding commit hashes directly into system builds is infeasible due to recursive commit dependencies and would break the reproducibility guarantees. Instead, this project develops a distributed service that gathers and correlates runtime system data across hundreds of servers.

\subsection{Nix and NixOS}
\subsection{Reproducible Builds and Store Paths}
\subsection{Store paths and Git commit hash relation}
% \subsection{Distributed Systems Essentials}
% \subsection{High Availability Concepts}
% \subsection{Existing Tools and Related Work}


\section{Overview of the Proposed System}
\subsection{Motivation and Rationale}
\subsection{Key Components} 
Here I will describe the three key components:

\begin{itemize}
    \item An activation that is deployed to all servers that are surveilled, that records all Nix activations and that also makes the data available
    \item A scraper that scrapes the data from the activation logger and pushes it to the server that displays the data
    \item Back end and front end for displaying data
\end{itemize}


\section{Design and Implementation}
\subsection{Requirements}
\subsubsection{Functional Requirements}
\subsubsection{Non-functional Requirements}

\subsection{System Architecture}
\subsubsection{Overall Architecture}
\subsubsection{Deployment Across Three Data Centres}
\subsubsection{API Design}
\subsubsection{Data Model}

\subsection{Implementation}
\subsubsection{Backend}
\subsubsection{Frontend}
\subsubsection{Integration with Build Pipeline}
\subsubsection{Security Considerations}


\section{Evaluation}
\subsection{Methodology}
\subsection{Reliability and Availability}
\subsection{Performance Evaluation}
\subsection{Failure Handling}

\section{Discussion}


\section{Conclusion and Future Work}
\subsection{Summary of Contributions}
\subsection{Limitations}
\subsection{Future Improvements}


\section{Reflection}
\subsection{Learning Outcomes}
\subsection{Engineering Considerations}


\section{Appendices}

