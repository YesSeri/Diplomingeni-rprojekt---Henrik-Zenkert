\documentclass[../main.tex]{subfiles}
\begin{document}

\section{Technical and Theoretical Background}
In modern distributed infrastructures, reproducibility and traceability are essential to ensure reliable operation across large fleets of servers. This project addresses that challenge in the context of NixOS-based systems deployed across multiple data centres.

The goal is to design and implement a host overview system that establishes a reliable mapping between Nix store paths and the git revisions from which they were built. This mapping is fundamental for two reasons:

1. Cluster consistency: Even though each server’s current-system path is unique, the mapping allows us to verify whether all nodes in a cluster correspond to the same git commit. This ensures that a distributed application is running the same software version across data centres.

2. Reproducibility verification: If multiple git commits produce identical nix store paths, we can confirm that changes in the source repository did not affect the system image of a given node. This provides strong guarantees about which changes are operationally relevant.

Embedding commit hashes directly into system builds is infeasible due to recursive commit dependencies and would break the reproducibility guarantees. Instead, this project develops a distributed service that gathers and correlates runtime system data across hundreds of servers.

\subsection{Nix and NixOS}
\subsection{Reproducible Builds and Store Paths}
\subsection{Store paths and Git commit hash relation}
% \subsection{Distributed Systems Essentials}
% \subsection{High Availability Concepts}
% \subsection{Existing Tools and Related Work}

\end{document}
